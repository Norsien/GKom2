\documentclass[a4paper,11pt]{article}

\usepackage[T1]{polski}
\usepackage[utf8]{inputenc}
\usepackage{amsmath}
\usepackage{graphicx}

\hoffset=-3.0cm                 
\textwidth=18cm                
\evensidemargin=0pt
\voffset=-3cm                  
\textheight=27cm                
\setlength{\parindent}{0pt}             
\setlength{\parskip}{\medskipamount}    
\raggedbottom                           

\title{Grafika komputerowa \\
{Zadanie 2}}
\author{Nowak Piotr, Smoliński Mateusz}
\date{maj 2022}

\begin{document}
\maketitle

\section{Realizacja zadania}
Zadanie zostało zrealizowane w środowisku python. Rozwiązanie 
jest oparte na programie napisanym do poprzedniego zadania. 
Zastosowany został algorytm skaningowy. Wierzchołki wielokątów
są punktami o współrzędnych określonch względem początku układu 
sceny. Obliczana jest ich pozycja względem kamery, następnie są 
one rzutowane na płaszczyznę i określane są krawędzie wielokątów. 
Program rysyje obraz poziomymi liniamii, od dołu. Kiedy przechodzi
przez krawędź sprawdza aktywne wielokąty i rysuje kolorem tego, 
który jest najblizej kamery.

\section{Obiekty}
Obiekty reprezentowane są za pomocą ścian, które są listami 
krawędzi. Krawędzie to pary wierzchołków, które mają usaloną 
pozycję wzgędem początku sceny. Każda ściana ma swój ustalony 
kolor, którym będzie rysowana.

\section{Zmienne kamery}
Podobnie jak w poprzednim zadaniu kamerę można przesuwać, 
obracać i zmieniać jej zakres widzenia (zoom).

\section{Rysowanie}
Obliczane są pozycje wierzchołków względem kamery. Wyznaczane są 
krawędzie i ich widoczność. Przypisywane są do wielokątów. 
Wszyskite krawędzie zapisane są na liście i są posortowane po 
najmniejszym Y. W pętli powtarzamy dla rosnącego Y rysowanie 
kkolejnych linii. Dodajemy nowe krawędzie do listy obecnych 
krawędzi i usuwamy te, które nie występują w danej linii. 
Sortujemy listę obecnych krawędzi po X. Przechodzimy po koleji 
po krawędziach. Każda krawędź aktywuje lub dezaktywuje wielokąt 
do którego należy. Jeżeli obecnie sprawdzana krawędź ma większy 
x niż poprzednia, to rysujemy pomiędzy tymi wartościami x poziomą 
linię kolorem wielokąta najbliżej kamery. Rozważany jest tylko 
wariant, że wielokąty nie mogą się przecinać.

\section{Dodatkowe uwagi}
Program nie działa gdy wielokąty się przecinają, ale możliwa by 
była odpowiednia modyfikacja.
Przesunięcie obrazu kamery, tak żeby był wycentrowany zostało 
wykonane na poziomie modułu rysującego. \\
\(\epsilon\) to wartość poniżej której obiekty traktujemy jako 
występujące za kamerą. Wynosi 0.000001. 
\end{document}
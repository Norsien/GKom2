\documentclass[a4paper,11pt]{article}

\usepackage[T1]{polski}
\usepackage[utf8]{inputenc}
\usepackage{amsmath}
\usepackage{graphicx}

\hoffset=-3.0cm                 
\textwidth=18cm                
\evensidemargin=0pt
\voffset=-3cm                  
\textheight=27cm                
\setlength{\parindent}{0pt}             
\setlength{\parskip}{\medskipamount}    
\raggedbottom                           

\title{Grafika komputerowa \\
{Zadanie 1}}
\author{Smoliński Mateusz}
\date{kwiecień 2022}

\begin{document}
\maketitle

\section{Realizacja zadania}
Zadanie zostało zrealizowane w środowisku python. Końce krawędzi 
są punktami o współrzędnych określonch wzgędem początku układu 
sceny. Obliczana jest ich pozycja względem kamery, następnie są 
one rzutowane na płaszczyzne i rysowane są odcinki miedzy 
uzyskanymi pozycjami. 

\section{Obiekty}
Obiekty reprezentowane są za pomocą wierzchołków zdefiniowanych jako 
wektor współrzędnych [x, y, z] i krawędzi zdefiniowanych jako pary
tych wierchołków (a, b). Różne obiekty są reprezentowane na ekranie 
za pomocą różnego koloru odcinków.

\section{Zmienne kamery}
Podczas działania programu kamerę można przesuwać, obracać i zmieniać 
zakres widzenia (zoom).

\subsection{Pozycja kamery}
Pozycja kamery reprezentowania jest jako wektor \(P\), podobnie jak 
wierzchołek.
\[
    P=
    \begin{bmatrix}
        c_x & c_y & c_z
    \end{bmatrix}    
\]
Możliwe jest przesunięcie pozycji kamery o 
ustaloną odległość wzdłóż jednej osi, obróbonych względem 
obecnej pozycji kamery. Aby przesunąć kamerę względem początku 
układu, w spossób taki, że względem kamery jest to przesunięcie 
w ustalonym kierunku należy rozwiązać równanie:
\[
    R \cdot X = 
    \begin{bmatrix}
        mv_x \\ mv_y \\ mv_z
    \end{bmatrix} 
\]
Gdzie \(R\) to macierz obrotu kamery, a \(mv\) to transponowany 
wektor przesunięcia, które chcemy otrzymać. Rozwiązując równanie
otrzymujemy \(X\), które dodajemy do \(P\).
\[
    X = R^{-1} \cdot
    \begin{bmatrix}
        mv_x \\ mv_y \\ mv_z
    \end{bmatrix}
    \quad
    P_{nowe} = P_{stare} + X^T
\]
\subsection{Obrót kamery}
Obrót kamery reprezentowany jest za pomocą macierzy \(R\), która 
zdefiniowana jest w następujący sposób:
\[
    R = 
    \begin{bmatrix}
        1 & 0 & 0 \\
        0 & cos(\alpha) & sin(\alpha) \\
        0 & -sin(\alpha) & cos(\alpha) 
    \end{bmatrix}\cdot
    \begin{bmatrix}
        cos(\beta) & 0 & -sin(\beta) \\
        0 & 1 & 0 \\
        sin(\beta) & 0 & cos(\beta) 
    \end{bmatrix}\cdot
    \begin{bmatrix}
        cos(\gamma) & sin(\gamma) & 0 \\
        -sin(\gamma) & cos(\gamma) & 0 \\
        0 & 0 & 1 
    \end{bmatrix}
\]
Gdzie \(\alpha\) jest obrotem wokół osi \(x\), 
\(\beta\) wokół osi \(y\) i \(\gamma\) wokół osi \(z\). Dzięki 
czemu dla zerowych wartości otrzymujemy początkową macierz:
\[
    R = 
    \begin{bmatrix}
        1 & 0 & 0 \\
        0 & 1 & 0 \\
        0 & 0 & 1 
    \end{bmatrix}
\]
Aby wykonać obrót względem obecnego obrotu zmieniamy \(R\) według 
sposobu:
\[
    R_{nowe} = R_{oborotu} \cdot R_{stare}
\]
Gdzie \(R_{oborotu}\) jest jedną z trzech macierzy użytych do 
wyznaczania \(R\) z usatlonym kątem obrotu.
\subsection{Zoom}
Zoom zmieniamy podmieniając osatią wartość z macierzy \(C\).
W prgramie moze tam wystąpić tylko wartość z listy.
\[
    C = 
    \begin{bmatrix}
        1 & 0 & 0 \\
        0 & 1 & 0 \\
        0 & 0 & zoom 
    \end{bmatrix}
\]
\section{Rzutowanie}
Punkty rzutujemy otrzymmując najpierw ich pozycję względem 
kamery, a nastepnie rzutując na płaszczyznę. Dla wierzchołów 
znajdujących się za kamerą zostały dodane specjalne warunki.
\[
    f = C \cdot R \cdot (P - A)^T
\]
gdzie A to wektor wspórzędnych wierchołka. Wtedy jeżeli \(f_z > 
\epsilon\) punkt rzutujemy i zapisujemy otrzymane współrzędne.
W przeciwym wypadku zostawiamy punkt tak jak jest i zaznaczamy 
że znajduje się za kamerą.
\section{Rysowanie}
Obiekty rysujemy biorąc po koleji każdą kolejną krawędź i 
sprawdzając gdzie znajdują się rzuty końców tych krawędzi.
Jeżeli oba są z przodu, to rysujemy obcinek między rzutami. 
Jeżeli oba są z tyłu, to znaczy że cała krawędź jest za kamerą 
i nie jest rysowane nic. W przypadku gdy jeden koniec występuje 
przed, a drugi za kamerą wyznaczamy odcinek między nimi w przestrzeni 
trójwymiarowej i przesuwamy wiechołek znajdujący się z tyłu o wektor 
bedący fragmentem tego odcinka, taki że będzie on przed kamerą i 
rzutujemy nowo otrzymany wierzchołek. Rysujemy obcinek między parą 
wierzchołków teraz, gdy oba są przed kamerą.

\section{Dodatkowe uwagi}
Przesunięcie obrazu kamery, tak żeby był wycentrowany zostało 
wykonane na poziomie modułu rysującego. \\
\(\epsilon\) to wartość poniżej której obiekty traktujemy jako 
występujące za kamerą. Wynosi 0.000001.
\end{document}